\section{Introduction}\label{sec:intro}

\textbf{What is machine learning}:
\begin{myitem}
    \item \textit{Arthur Samuel (1959)}: “Machine Learning: Field of study that gives computers the ability to learn without being explicitly programmed”;
    \item \textit{Tom Mitchell (1998)}: “Well posed Learning Problem: A computer program is said to learn from experience $E$ with respect to some task $T$ and some performance measure $P$, if its performance on $T$, as measured by $P$, improves with experience $E$”;
    \item \textit{Examples}: database mining, applications which cannot be programmed by hand, self customizing programs, understanding human learning.
\end{myitem}

\textbf{What is Perception (computer vision)}:
\begin{myitem}
    \item \textit{Science}: understand how do we see (explore computational model of human vision);
    \item \textit{Engineering}: build systems that perceive the world;
    \item \textit{Applications}: medical imaging, surveillance, entertainment, graphics, car industry.
\end{myitem}

The three keys to successes in \textbf{Deep Learning} are algorithms, data, computation.

\textbf{Basic concepts of Computer Vision}:
\begin{myitem}
    \item Goals: recognize object, perceive depth...;
    \item \textit{Recognition} problems:
    \begin{itemize}
        \item \textit{Identification}: recognize a specific object (es: your pen),
        \item \textit{Classification}: recognize a class of objects (es: any pen), it's also called \textit{generic object recognition} or \textit{object categorization};
    \end{itemize}
    \item \textit{Segmentation}: separate pixels belonging to the foreground (object) and the background;
    \item \textit{Localization/Detection}: position of the object in the scene, pose estimate;
    \item Challenges: multi scale, multi view, multi class, varying illumination, occlusion, cluttered background, articulation, high intraclass variance, low interclass variance.
\end{myitem}
